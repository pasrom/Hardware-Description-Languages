\documentclass[./\jobname.tex]{subfiles}
\begin{document}
%
\def\codeFolderName{debounce}
\def\codeFolderNameB{}
%
\chapter{Debouncer}
%
\section{Einleitung}
%
Der Switch, der für den Betrieb der Stoppuhr bestimmt ist, muss entprellt werden, da ansonsten die Statemachine nicht ordnungsgemäß funktioniert. Dieses Beispiel entprellt die \enquote{high} sowie \enquote{low} Flanke.
%
\begin{itemize}
	\item Wenn der Switch \enquote{low} ist, wird der erste Counter gestartet von einem Initialwert (alle Bits auf eins). Das Signal ist solange gültig, bis der Counter null erreicht hat. Wenn der Switch \enquote{high} schaltet, so wird der erste Counter sofort rückgesetzt auf seinen Initialwert.
	\item Wenn der Switch \enquote{high} ist, wird der zweite Counter gestartet von einem Initialwert (alle Bits auf eins). Das Signal ist solange gültig, bis der Counter null erreicht hat. Wenn der Switch \enquote{low} schaltet, so wird der zweite Counter sofort rückgesetzt auf seinen Initialwert.
	\item Die Bitbreite des Counters muss in einem Header File mit einem Makro (\enquote{BITS}) definiert werden. Für diese Implementierung wird \enquote{BITS} auf drei gesetzt.
	\item Die Flankenerkennung der fallenden und steigenden Flanke des entprellten Signals wird als \enquote{sw\_hi} und \enquote{sw\_lo} ausgegeben. Beide Ausgänge sind nur für einen Clock Zyklus auf \enquote{high}.
\end{itemize}
%
\section{Implementierung}
%
Die Implementierung erfolgt wie in \autoref{lst:counter_updn} dargestellt ist.\par
%
In Codezeile \autoref{code:ff sw_hi_cnt begin} bis \autoref{code:ff sw_hi_cnt end} ist der Counter für die positive Flanke in einem Flipflop dargestellt. In Codezeile \autoref{code:ff sw_lo_cnt begin} bis \autoref{code:ff sw_lo_cnt end} ist der Counter für die negative Flanke in einem Flipflop dargestellt.\par
%
Um den Übergang der Counter auf null zu erkennen wurde jeweils für die zwei Counter ein eigenes Flipflop erstellt, dies ist in Codezeile \autoref{code: edge up begin} bis \autoref{code: edge up end} (positive Flanke) und in Codezeile \autoref{code: edge down begin} bis \autoref{code: edge down end} (negative Flanke) dargestellt.\par
%
Für den entprellten Ausgang wurde ebenfalls ein eigenes Flipflop implementiert, dies ist in Codezeile \autoref{code: ff output begin} bis \autoref{code: ff output end} zu sehen.\par
%
Die Ausgänge \enquote{sw\_lo\_cnt\_zero} und \enquote{sw\_hi\_cnt\_zero} werden mit den jeweiligen Countern zugewiesen, wenn der Counter null ist, so ist der Ausgang eins.
%
\lstinputlisting[language=Verilog,label={lst:counter_updn}, caption=Implementierung]{./../code/\codeFolderName\codeFolderNameB/src/\codeFolderName.sv}
%
\section{Test Bench}
%
In \autoref{lst:tb_counter_updn} ist die Testbench ersichtlich.
%
\lstinputlisting[language=Verilog,label={lst:tb_counter_updn}, caption=Testbench für den Up / Down Counter]{./../code/\codeFolderName\codeFolderNameB/sim/tb_\codeFolderName.sv}
%
\section{Simulationsscript}
%
In \autoref{lst:sim_tb_counter_updn} ist das Simulationsscript dargestellt. Es beinhaltet dieselben Befehle wie in der letzten Lehrveranstaltung, natürlich angepasst an den Debouncer.
%
\lstinputlisting[language=tcl,label={lst:sim_tb_counter_updn}, caption=Simulationsscript]{./../code/\codeFolderName\codeFolderNameB/sim/sim_tb_\codeFolderName.tcl}
%
\section{Transkript und Waveform Window}
%
In \autoref{fig: wavewindow_up_down} ist das Waveform Window dargestellt. Es zeigt die geforderten Testfälle. Wie zu sehen ist, wurden die Anforderungen erfüllt.
%
\begin{figure}[H]
	\centering
	\noindent\adjustbox{max width=\textwidth}{%falls größer als \textwidth, wird das Bild verkleinert
		%trim option's parameter order: left bottom right top
	\includegraphics[width=1\textwidth]{./../code/\codeFolderName\codeFolderNameB/doc/wavewindow_\codeFolderName.PNG}
	}
	\unterschrift{Waveform Window}{eigene Ausarbeitung}{}
	\label{fig: wavewindow_up_down}
\end{figure}
%
%In \autoref{ausgabeTwo} ist der Output des Testcripts zu sehen. Alle Testfälle wurden bestanden.
%%
%\lstinputlisting[language=tcl,
%label={ausgabeTwo}, caption=Commandline Output A]{./../code/counter_updn/doc/console_output.txt}
%
%\section{Vor- und Nachteile der Implementierung}
%%
%Folgend sind die Vor- und Nachteile der Implementierung gelistet:
%%
%\begin{description}
%	\item[Vorteile] \hfil
%	\begin{itemize}
%		\item Variable Bitlänge \pfeil Entprelldauer kann angepasst werden.
%	\end{itemize}
%	\item[Nachteile] \hfil
%	\begin{itemize}
%		\item Die Clock Frequenz darf nur so hoch gewählt werden, wie es das \enquote{Back Propagation Delay} erlaubt. Somit ist die maximale Zählfrequenz begrenzt.
%		\item Bei Anwahl des Down-Bits sollte beim Reseten der Counter Wert nicht auf \enquote{0} initalisiert werden, sondern mit dem maximalen Wert
%		\begin{align}
%		2~<<~ (`counterSize-1) - 1
%		\end{align}
%		\item Ebenfalls wäre es gut, den Startwert des Counters selbst wählen zu können.
%	\end{itemize}
%\end{description}
%
\end{document}
